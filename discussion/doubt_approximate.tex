\documentclass[12pt]{article}
\usepackage[margin=1in]{geometry}
\usepackage{bm}
\usepackage{amsmath,amsthm,amssymb,amsfonts}

\begin{document}

\title{Doubt on the approximation of PV formulation}
\author{Li Chiqin}
\maketitle

The definition we use in PV calculation is
\begin{equation}
    q = -g\eta\frac{\partial{\theta}}{\partial{p}} + g(\frac{\partial{v}}{\partial{p}}
    \frac{\partial{\theta}}{\partial{x}} - \frac{\partial{u}}{\partial{p}} \frac{\partial{\theta}}{\partial{y}})
\end{equation}

From the definition of PV in isentropic coordinate
\begin{equation}
    P \equiv (\zeta_{\theta} + f)(-g \frac{\partial{\theta}}{\partial{p}})
\end{equation}
and we use the coordinate transform
\begin{gather}
    \Big(\frac{\partial{u}}{\partial{y}}\Big)_{\theta} = \Big(\frac{\partial{u}}{\partial{y}}\Big)_{p}
    + \Big(\frac{\partial{u}}{\partial{p}}\Big) \Big(\frac{\partial{p}}{\partial{y}}\Big)_{\theta}\\
    \Big(\frac{\partial{v}}{\partial{x}}\Big)_{\theta} = \Big(\frac{\partial{v}}{\partial{x}}\Big)_{p}
    + \Big(\frac{\partial{v}}{\partial{p}}\Big) \Big(\frac{\partial{p}}{\partial{x}}\Big)_{\theta}
\end{gather}
substituting (3),(4) into (2), in order to get the form of (1), we need to establish equation like:
\begin{equation}
    \frac{\partial{v}}{\partial{p}} \Big(\frac{\partial{p}}{\partial{x}}\Big)_{\theta}
    \frac{\partial{\theta}}{\partial{p}} \equiv \frac{\partial{v}}{\partial{p}} \Big(\frac{\partial{\theta}}{\partial{x}}\Big)_{p}
\end{equation}
However, since the isobar and isentropic line are not parellel, the identity should not be accurately right,
is this where the approximation of PV take place? If so, the PV should not be conservative in accurate sense?\\


On the other hand, I have no idea why $f_{0}\frac{\partial{\Psi}}{\partial{\pi}} = -\theta$. Can you explain it further?
And when we nondimensionalize $\Psi\ and \ \Phi$ using Wang and Zhang 2004, are we suppose Rossby Number about 1 so that
$U \approx f_{0}L$ and then $\Phi \approx f_{0}LU$?





\end{document}
